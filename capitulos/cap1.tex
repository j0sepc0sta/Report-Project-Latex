\chapter{Introdução}\label{cap1}

Este documento pretende guiar o Estudante na elaboração do relatório de Projeto/Estágio, do 3º ano da \gls{ESTIG} do \gls{IPB} a frequentar o curso de \gls{EEC}.\par
O autor deverá ter em consideração as seguintes regras gerais na elaboração do documento:
\begin{itemize}
	\item O documento deve ser redigido em português ou inglês com um estilo adequado e correto do ponto de vista gramatical (quer do ponto de vista sintático quer semântico);
	\item Ter especial cuidado com o uso de adjetivos (facilmente conduzem ao exagero), advérbios (nada, ou quase nada, acrescentam) e sinais de pontuação (em especial o uso correto das vírgulas);
	\item O estilo adotado para a redação deve ser coerente com as exigências de um trabalho científico encontrado em publicações impressas;
	\item De uma forma genérica deve usar a 3ª pessoa do singular (eventualmente do plural), exceção feita aos locais onde tal \'e claramente desajustado, por exemplo, na secção dos agradecimentos;
	\item Usar o estilo \textit{it\'{a}lico} sempre que s\~ao utilizados termos em l{\'i}nguas diferentes da l{\'i}ngua adotada no relat{\'o}rio, para escrever símbolos matemáticos, por exemplo $\omega$, ou \textit{$\omega$};
	\item O uso de acr\'onimos implica que na 1ª vez que são utilizados se apresentem por extenso, colocando entre parênteses a respetiva sigla que se passará a usar. Todos os acrónimos devem ser apresentados por ordem alfabética na secção ``Lista de Acrónimos'';
    	\item O uso correto de unidades, seus múltiplos e submúltiplos \footnote{As unidades utilizadas ao longo do relatório deverá ser introduzida em \textbf{Nomenclatura}.};
	\item As imagens e tabelas devem, por princípio, aparecer no topo ou no fundo da página. A legendas  das figuras surgem imediatamente após as figuras e no caso das tabelas as legendas antecedem as mesmas;
	\item Todas as figuras, tabelas e restantes listagens devem ser mencionadas no texto por forma a que fiquem enquadradas nas ideias transmitidas pelo autor. Esta referência, regra geral, deverá ser feita antes da ocorrência da figura, tabela ou listagem;
	\item Indicar ao longo do texto as referências documentais usadas, em especial nas citações (puras ou literais), assinaladas com a utilização de aspas, como também no caso de reutilização de gráficos, figuras, tabelas, fórmulas, etc., de outras fontes;
\end{itemize}

De uma forma já mais específica, neste primeiro capítulo obrigatório (``Introdução'') o autor deve \footnote{Recomenda-se para cada item a utilização de uma secção}:
\begin{itemize}
	\item Contextualizar a proposta de trabalho no âmbito da empresa, de um outro trabalho já realizado, do ponto de vista científico e/ou tecnológico, etc.;
	\item Apresentar de forma clara os objetivos que se propõe atingir;
	\item Descrever de forma sucinta, mas objetiva, a solução preconizada ou a hipótese colocada;
	\item Apresentar de forma resumida, mas clara, os desenvolvimentos efetuados;
	\item Identificar como foi validada e avaliada a solução encontrada;
	\item Descrever a organização do documento.
\end{itemize}
