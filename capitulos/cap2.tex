\chapter{Contexto e Tecnologias/Ferramentas}\label{cap2}

\begin{adjustwidth}{2.5cm}{0.3cm}
Nunca colocar uma $section$ depois de um $chapter$ por isso, aqui deve-se fazer um breve resumo do que se vai falar ao longo do capítulo (2 - 3 linhas), por exemplo....
Neste capítulo será abordado formas de incluir figuras, tabelas e equações.
\end{adjustwidth}

\section{Figuras}
Nesta secção será abordado como poderá-se colocar um figura num documento $Latex$.\par
De acordo com \cite{overleaf}, em $Latex$ Básico que para incluir figuras é necessário o pacote $graphicx$, que já está introduzido no ficheiro `` package.text''. Posto isto, ao longo do capítulo é importante referir o significado da figura, como por exemplo ``Na Figura \ref{fig:leon} será ilustrado um exemplo de uma figura''. Em segundo a legenda de uma figura fica  \underline{sempre} depois da figura.
\begin{figure}[htpb]
    \centering
    \includegraphics[scale=0.4]{imagens/lion_large.png}
    \caption{Exemplo de uma figura.}
    \label{fig:leon}
\end{figure}\par 
Por vezes é necessário colocar 2 figuras simumltaneamente como será ilustrada na figura \ref{fig:leon1}
\begin{figure}[H]
    \centering
    \subfigure[Figura 1]{\label{plot1}
    \includegraphics[scale=0.4]{imagens/lion_large.png}
    }\hspace{.5cm}
    \subfigure[Figura 2]{\label{plot2}
    \includegraphics[scale=0.4]{imagens/lion_large.png}
    }
    \caption{Figuras apresentadas com o pacote $subfigure$.}
    \label{fig:leon1}
\end{figure}\par
Na Figura \ref{fig:leon1} cada sub-figura têm uma sub-legenda, na Figura \ref{fig:_lado_a_lado} será ilustrado duas Figura com apenas uma legenda.
\begin{figure}[H]
    \centering
    \includegraphics[scale=0.4]{imagens/lion_large.png} \ \ \ \ \ \ \
    \includegraphics[scale=0.4]{imagens/lion_large.png}
    \caption{Exemplo de duas figuras, uma ao lado da outra.}
    \label{fig:_lado_a_lado}
\end{figure}

\section{Tabelas}
Nesta secção será abordado como poderá-se colocar um tabela num documento $Latex$.\par
Segundo \cite{overleaftables}, uma tabela é definida entre os comandos \verb|\begin{tabular}| e \verb|\end{tabular}|, já seguir será ilustrado um exemplo.\par
\begin{table}[H]
    \renewcommand{\arraystretch}{1.5}
    \centering
    \caption{Tabela centralizada.}
    \label{tab1}
    \begin{tabular}{ccc}
        \hline
        Coluna  & Coluna  & Coluna \\
        \hline
        a & b & c \\
        d & e & f \\
        \hline
    \end{tabular}
\end{table}\par
Após \verb|\begin{tabular}| é colocado, entre \verb|{}|, ccc, o que indica que a tabela terá 3 colunas, todas centralizadas. O número de letras indica o número de colunas e a letra o seu alinhamento: 
\begin{itemize}
    \item c para colunas com texto alinhado centralizado;
    \item l para colunas com texto alinhado à esquerda;
    \item r para colunas com texto alinhado à direita.
\end{itemize}\par
Para indicar uma separação de coluna use-se \verb|&|. Para indicar o número linhas usa-se duas
barras juntas, \verb|\\|, o que significa quebra de linha. O comando \verb|\hline| é responsável por colocar uma linha horizontal na tabela e o comando \verb|\cline{-}| faz uma linha horizontal somente entre as colunas indicadas. Para inserir linhas verticais usa-se \verb||| entre as letras que indicam o alinhamento da coluna.
\begin{table}[H]
    \renewcommand{\arraystretch}{1.5}
    \centering
    \caption{Tabela com alinhamento à esquerda.}
    \label{tab2}
    \begin{tabular}{|l|cc|}
    \hline
    Coluna & Coluna  & Coluna \\
    \hline \hline
    A & B & C \\
    \cline{2-3}
    D & E & F \\
    \hline
    \end{tabular}
\end{table}\par 
Se uma coluna receber um texto longo e seja necessário que haja uma quebra de linha dentro da célula, em vez de usar as letras c, l ou r usa-se \verb|p{}|, onde dentro \verb|{}| incluiu-se o tamanho da linha.
\begin{table}[H]
    \renewcommand{\arraystretch}{1.5}
    \centering
    \caption{Tabela usando $p\{\}$.}
    \label{tab3}
    \begin{tabular}{ccp{5cm}}
    \hline
    C & C & Coluna de Texto \\
    \hline
    A & B & Aqui será digitado um texto grande,
    mas a largura da célula é fixa em 5 \si{\cm}.\\
\hline
\end{tabular}
\end{table}\par
É possível tornar as tabelas mais bonitas, para isso é necessário  usar \verb|\usepackage{booktabs}|, ou seja este pacote retira o \verb|\hline| e coloca: 
\begin{itemize}
    \item \verb|\toprule|, para a linha superior da tabela;
    \item \verb|\midrule|, para as linhas no meio da tabela;
    \item \verb|\bottomrule|, para a linha abaixo da tabela.
\end{itemize}
\begin{table}[H]
    \renewcommand{\arraystretch}{1.15}
    \centering
    \caption{Tabela usando o pacote $booktabs$.}
    \label{tabela3}
    \begin{tabular}{llr}
        \toprule
        \multicolumn{2}{c}{Item} \\
        \cmidrule(r){1-2}
        Animal & Description & Price (\$)\\ \midrule
        Gnat & per gram & 13.65 \\
        & each & 0.01 \\
        Gnu & stuffed & 92.50 \\
        Emu & stuffed & 33.33 \\
        Armadillo & frozen & 8.99 \\
        \bottomrule
    \end{tabular}
\end{table}


\section{Equações}
Em qualquer fórmula matemática existem constantes e variáveis, o $Latex$ adota como convenção de
trabalho, modificar a fonte e a apresentação dos elementos em função do seu tipo, constante ou variável, como por exemplo  $p''=max\{f(y),g(x)\}$ \footnote{Sempre que iniciar uma equação é obrigatório ter legenda ``Eq.2.1''.}.
\begin{equation}
    f_X(x) = \frac{1}{\sqrt{2 \pi \sigma^2}}e^{-\frac{(x-\mu)^2}{2\sigma^2}}
\end{equation}
\begin{equation*}
    f_X(x) = \frac{1}{\sqrt{2 \pi \sigma^2}}e^{-\frac{(x-\mu)^2}{2\sigma^2}}
\end{equation*}
\begin{center}
    $f_X(x) = \frac{1}{\sqrt{2 \pi \sigma^2}}e^{-\frac{(x-\mu)^2}{2\sigma^2}}$
\end{center}\par
Para mais informações \cite{overleafsimbolos,simbolos}.
